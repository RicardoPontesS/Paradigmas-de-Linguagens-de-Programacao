% Prof. Dr. Ausberto S. Castro Vera
% UENF - CCT - LCMAT - Curso de Ci\^{e}ncia da Computa\c{c}\~{a}o
% Campos, RJ,  2022
% Disciplina: Paradigmas de Linguagens de Programa\c{c}\~{a}o
% Aluno: Ricardo Willian Pontes da Silva


\chapter{Conclus\~{o}es}

Ao decorrer do trabalho, foram estudadas e abordadas diversas características internas e marginais da linguagem Python com o objetivo de elucidar primordialmente seu paradigma de linguagem de programação. Para isso, foi utilizado de plataformas como google acadêmico, scielo, e livros didáticos disponíveis nas referências do presente trabalho. Como metodologia de pesquisa foi utilizado da abordagem qualitativa e exploratória através da literatura disponível.\\
As principais dificuldades enfrentas ao decorrer da pesquisa foi o período de adaptação com a sintaxe da linguagem Tex, A escassez de pesquisas científicas recentes dos assuntos abordados e também o período de familiaridade com a linguagem Python propriamente dito, uma vez que se tornou necessário a criação de códigos próprios para o enriquecimento do trabalho. Porém, esses empecilhos foram devidamente superados e implicitamente contribuíram para o enriquecimento do conteúdo do artigo, mas também no conhecimento do autor.\\
O artigo acima foi construído de modo que pudesse se tornar um guia para novos usuários que desejam ingressar na linguagem Python e para desenvolvedores experientes que busquem tópicos específicos da linguagem. Foi buscado também a abordagem do assunto de uma maneira clara e simples. Foi foco do presente trabalho, também, a abordagem implícita de técnicas de construção textual e normas regulamentadoras.\\
Como indicações de pesquisas futuras, é recomendado o aprofundamento na sintaxe da linguagem Python, o uso da linguagem no âmbito da inteligência artificial e por fim, uma maior elucidação em conceitos de gerenciamento de memória. 


   \begin{figure}[H]
    \begin{center}
        \caption{Linguagens de programa\c{c}\~{a}o modernas} \label{ling2}
        \includegraphics[width=12cm]{Python02.png} \\
        {\tiny \sf Fonte: O autor }
    \end{center}
   \end{figure} 