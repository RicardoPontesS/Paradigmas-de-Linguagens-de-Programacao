% Prof. Dr. Ausberto S. Castro Vera
% UENF - CCT - LCMAT - Curso de Ci\^{e}ncia da Computa\c{c}\~{a}o
% Campos, RJ,  2022
% Disciplina: Paradigmas de Linguagens de Programa\c{c}\~{a}o
% Aluno: Ricardo Willian Pontes da Silva


\chapter{ Programa\c{c}\~{a}o Orientada a Objetos com Python}
O paradigma Orientado a Objetos é um conjunto de padrões exercidos de modo a solucionar um determinado problema, classificando-o como um objeto do mundo real. Como já citado anteriormente no tópico "Orientação a objetos"  na página \pageref{ling2} baseado em \cite{FBarelli2019} , em Python, toda e qualquer construção da linguagem é considerada um objeto. Desta forma, mesmo quando não se usa o paradigma POO, ainda assim se utiliza de objetos.
   %%%%%%%%======================
    \section{Classes}
Como citado por \cite{Perkovic2016},em Python, uma classe pode ser entendida como um "agrupamento" de métodos e atributos de um determinado objeto. Vale ressaltar também que a criação da classe deverá ocorrer antes da criação do seu respectivo objeto. Abaixo veremos o exemplo da utilização de uma classe na linguagem Python.

   \begin{lstlisting}
   class MinhaClasse:
   
    num = 5

    \end{lstlisting}
\section{Objetos}
Desta forma, seguindo a mesma lógica de classe, que busca implementar o mundo real e suas soluções de problemas no conceito de algoritmos computacionais, os objetos podem ser definidos como qualquer existência abstrata de modo que possua comportamentos e características. Abaixo será mostrado um exemplo prático da criação de um objeto na linguagem Python.

   \begin{lstlisting}   	
   obj = MinhaClasse()
   soma = obj.num + 10
   print(soma)

\end{lstlisting}
Outra funcionalidade das classes contida na linguagem Python é o que se entende como construtor de uma classe no conceito de POO. Esse comando é utilizado para atribuir valores nos determinados atributos que o objeto está modificando. Abaixo segue um pequeno exemplo de como utilizar a palavra reservada init().

   \begin{lstlisting} 
   	  	
   class Pessoa:
   def __init__(self, nome, idade):
   self.nome = nome
   self.idade = idade
   
   Joao = Pessoa("Joao", 25)
\end{lstlisting}

   %%%%%%%%======================
    \section{Operadores ou M\'{e}todos}
Em suma, podemos classificar um método como sendo um uma função ligada a um objeto instância. Os métodos podem ser de classes definidas pelo próprio usuário, mas também métodos já nativos da própria linguagem. Para se declarar um método em Python, é necessário utilizar da palavra reservada def, seguida de parênteses, que possuem a função de agrupar os parâmetros e por fim, o uso também da palavra reservada init, a qual já foi detalhada anteriormente. Abaixo será apresentado um pequeno exemplo da declaração de um método na linguagem Python.
 
   \begin{lstlisting} 
	
   class Pessoa:
   def __init__(self, nome, idade):
   self.nome = nome
   self.idade = idade

   def setNome(self, nome):
   self.nome = nome

   def setIdade(self, idade):
   self.idade = idade

\end{lstlisting}


Vale destacar também que, há uma ligeira diferença entre métodos e funções na linguagem Python. Como já citado no parágrafo anterior, um método está ligado a um objeto, que por sua vez está ligado em uma classe. Já uma função é não está definida dentro de uma classe, portanto, não pertence a um objeto de instância.
   %%%%%%%%======================
    \section{Heran\c{c}a}
    %%%%%%%%======================
No conceito do paradigma orientado a objetos, herança é a capacidade pela qual uma linguagem tem de estender funcionalidades de uma classe, ou seja, podemos entender a herança como sendo um tipo de relacionamento entre classes. O uso de herança se faz necessário uma vez que possibilita a reutilização de códigos e também representa de maneira satisfatória o conceito do mundo real em algoritmos computacionais. Abaixo será expressado um simples exemplo da utilização de herança na linguagem Python.
   \begin{lstlisting} 	

class Pessoa:
	def __init__(self, nome, idade):
	self.nome = nome
		self.idade = idade

class Estudante(Pessoa):
	pass

Ricardo = Estudante ('Ricardo Willian Pontes da Silva',21)
\end{lstlisting}

Vale destacar o uso da palavra reservada pass na classe Estudante que está estendendo a classe Pessoa,que tem como objetivo herdar atributos e métodos na classe mãe sem incluir seus específicos. 
   %%%%%%%%======================
    \section{Estudo de Caso: }
    %%%%%%%%====================== 