% Prof. Dr. Ausberto S. Castro Vera
% UENF - CCT - LCMAT - Curso de Ci\^{e}ncia da Computa\c{c}\~{a}o
% Campos, RJ,  2022
% Disciplina: Paradigmas de Linguagens de Programa\c{c}\~{a}o
% Aluno: Ricardo Willian Pontes da Silva


\chapter{ Programa\c{c}\~{a}o Orientada a Objetos com Python}
O paradigma Orientado a Objetos é um conjunto de padrões exercidos de modo a solucionar um determinado problema, classificando-o como um objeto do mundo real. Como já citado anteriormente no tópico "Orientação a objetos"  na página \pageref{ling2} baseado em \cite{FBarelli2019} , em Python, toda e qualquer construção da linguagem é considerada um objeto. Desta forma, mesmo quando não se usa o paradigma POO, ainda assim se utiliza de objetos.
   %%%%%%%%======================
    \section{Classes e Objetos}
Como citado por \cite{Perkovic2016},em Python, uma classe pode ser entendida como um "agrupamento" de métodos e atributos de um determinado objeto. Vale ressaltar também que a criação da classe deverá ocorrer antes da criação do seu respectivo objeto. Abaixo veremos o exemplo da utilização de uma classe na linguagem Python.

   \begin{lstlisting}
   class MinhaClasse:
   
    num = 5

    \end{lstlisting}
Desta forma, seguindo a mesma lógica de classe, que busca implementar o mundo real e suas soluções de problemas no conceito de algoritmos computacionais, os objetos podem ser definidos como qualquer existência abstrata de modo que possua comportamentos e características. Abaixo será mostrado um exemplo prático da criação de um objeto na linguagem Python.

   \begin{lstlisting}
	obj = MinhaClasse()
	soma = obj.num + 10
	printf(soma)
\end{lstlisting}
  
   %%%%%%%%======================
    \section{Operadores ou M\'{e}todos}
    %%%%%%%%======================


   %%%%%%%%======================
    \section{Heran\c{c}a}
    %%%%%%%%======================


   %%%%%%%%======================
    \section{Estudo de Caso: }
    %%%%%%%%====================== 