% Prof. Dr. Ausberto S. Castro Vera
% UENF - CCT - LCMAT - Curso de Ci\^{e}ncia da Computa\c{c}\~{a}o
% Campos, RJ,  2022
% Disciplina: Paradigmas de Linguagens de Programa\c{c}\~{a}o
% Aluno: Ricardo Willian Pontes da Silva


\chapter{ Programa\c{c}\~{a}o Orientada a Objetos com Python}
O paradigma Orientado a Objetos é um conjunto de padrões exercidos de modo a solucionar um determinado problema, classificando-o como um objeto do mundo real. Como já citado anteriormente no tópico "Orientação a objetos"  na página \pageref{ling2} baseado em \cite{FBarelli2019} , em Python, toda e qualquer construção da linguagem é considerada um objeto. 
   %%%%%%%%======================
    \section{Classes e Objetos}
    %%%%%%%%======================


   \begin{lstlisting}
    class NomeClasse:

    def metodo1:

    def Metodo2:

    \end{lstlisting}

   %%%%%%%%======================
    \section{Operadores ou M\'{e}todos}
    %%%%%%%%======================


   %%%%%%%%======================
    \section{Heran\c{c}a}
    %%%%%%%%======================


   %%%%%%%%======================
    \section{Estudo de Caso: }
    %%%%%%%%====================== 