% Prof. Dr. Ausberto S. Castro Vera
% UENF - CCT - LCMAT - Curso de Ci\^{e}ncia da Computa\c{c}\~{a}o
% Campos, RJ,  2022
% Disciplina: Paradigmas de Linguagens de Programa\c{c}\~{a}o
% Aluno: Ricardo Willian Pontes da Silva


\chapter{Ferramentas existentes e utilizadas}
Neste capítulo será apresentadas algumas ferramentas para auxiliar no desenvolvimento utilizando a linguagem de programação Python, algumas dessas ferramentas foram utilizadas até mesmo para realizar esse trabalho. As ferramentas a seguir terão as \begin{itemize}
	\item Nome da ferramenta
	\item Versão atual e utilizada
	\item Descrição e informações
	\item Telas capturadas da ferramenta
	\item Endereço na Internet
\end{itemize}


    \section{Visual Studio }

O Visual Studio é um ambiente de desenvolvimento poderoso para a linguagem Python por meio dos seus recursos e cargas de trabalho de ciência de dados. Com a utilização do Visual Studio para a criação de códigos com a linguagem Python a aprendizagem se torna de fácil acesso e gratuita com diversas bibliotecas disponíveis no seu ambiente. Com a união dessas ferramentas, é possível criar aplicativos Web, serviços Web, aplicativos de desktop, scripts e computação científica no Visual Studio, podendo ser utilizados por cientistas e desenvolvedores profissionais e casuais.

    \section{Compilador XYZ}


    \section{Interpretador UVW}


    \section{Ambientes de Programa\c{c}\~{a}o IDE MNP} 