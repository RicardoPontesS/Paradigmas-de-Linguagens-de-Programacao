% Prof. Dr. Ausberto S. Castro Vera
% UENF - CCT - LCMAT - Curso de Ci\^{e}ncia da Computa\c{c}\~{a}o
% Campos, RJ,  2022
% Disciplina: Paradigmas de Linguagens de Programa\c{c}\~{a}o
% Aluno: Ricardo Willian Pontes da Silva


\chapter{ Conceitos b\'{a}sicos da Linguagem Python}

Neste capítulo será apresentado conceitos básicos da linguagem Python, como, seus tipos primitivos de variáveis e constantes presentes na linguagem, suas estruturas, sua sintaxe e semântica e também seu o uso de pacotes disponíveis para a construção de códigos. Os livros b\'{a}sicos para o estudo da Linguagem Python s\~{a}o:\cite{Manzano2018} ,\cite{Summerfield2013}, \cite{Guttag2015}, \cite{Perkovic2016}


Considerando que a linguagem Python (\cite{Manzano2018}) \'{e} definida como sendo pseudocódigo executável pela sua semelhança com a facilidade de entendimento de algoritmos voltados para a aprendizagem. Desta forma, nos sentiremos familiarizados com a construção de seus códigos, ou então não ocorrerá grandes dificuldades de serem entendidos.
    %%%%%%%%=================================
    \section{Vari\'{a}veis e constantes}
Antes de entendermos o que se refere a parte prática da aplicação de variáveis e constantes na linguagem Python, vale ressaltar uma nota que já foi expresso anteriormente, que se trata do fato de que nesta linguagem, tudo é classificado como objeto. Desta forma, conceitos anteriormente conhecidos de linguagens anteriores não serão mais utilizados, como a direta referência de memoria ao ocorrer a criação e alocação de variáveis.\\
Com isso, é dito que a linguagem Python é dinâmicamente tipada, o que refere ao fato de não se precisar especificar o seu tipo primitivo "tamanho que ocuparia na memória".  


    %%%%%%%%=================================
    \section{Tipos de Dados B\'{a}sicos}
 De acordo com, \cite{Menezes2016}, em Python temos variáveis de tipos primitivos como \textit{int}, que alocará números inteiro entre -2.147.483.648 a 2.147.483.647, ou seja, esse tipo de variável tem como tamanho na memória 32 bits. Temos também o tipo \textit{float}, que tem como capacidade armazenar números facionários entre $1.7*10^{-8}$ a $3.4*10^{8}$. E por fim, temos também o tipo primitivo \textit{string}, que terá seu tamanho na memória variável de acordo com sua alocação. 

\subsection{Inteiro (int)}
Dizemos que um valor atribuído a uma variável é do tipo \textit{int} se e somente se, o mesmo não houver pontos flutuantes. Abaixo está sendo representado um exemplo de atribuição na linguagem Python para um melhor entendimento.

\begin{lstlisting}
	>>> #atribuicao simples em Python
	>>> int numero = 2
	>>> int numero =+ 4
	O valor final da vari\'{a}vel numero ser\'{a} 6, 
	uma vez que a mesma recebeu duas atribuicoes.
\end{lstlisting} 


     %%%........................
            \subsection{String}
     %%%........................
            Um string \'{e} uma sequ\^{e}ncia de caracteres considerado como um item de dado simples. Para Python, um string \'{e} um array de caracteres ou qualquer grupo de caracteres escritos entre doble aspas ou aspas simples. Por exemplo,
    \begin{lstlisting}
    >>> #usando aspas simples
    >>> pyStr1 = 'Brasil'
    >>> print (pyStr1)  Brasil
    >>> #usando aspas duplas
    >>> pyStr2 = "Oi, tudo bem?"
    >>> print (pyStr2)
    Oi, tudo bem?
    \end{lstlisting}

    \begin{itemize}
      \item \textit{Concatena\c{c}\~{a}o de strings}\\
            Strings podem ser concatenadas utilizando o operador +, e o seu comprimento pode ser calculado utilizado o operador \texttt{len(string)}
     \begin{lstlisting}
    >>> # concatenando 2 strings
    >>> pyStr = "Brasil" + " verde amarelo"
    >>> print (pyStr)
    Brasil verde amarelo
    >>> print (len(pyStr))
    20
    \end{lstlisting}

      \item \textit{Operador de indexa\c{c}\~{a}o}\\
      Qualquer caracter de um string o sequ\^{e}ncia de caracteres pode ser obtido utilizando o operador de indexa\c{c}\~{a}o []. Existem duas formas de indexar em Python, os caracteres de um string:\\
      \begin{description}
        \item[Index com inteiros positivos] indexando a partir da esquerda come\c{c}ando com 0 e  onde 0 \'{e} o index do primeiro caracter da sequ\^{e}ncia
        \item[Index com inteiros negativos] indexando a partir da direita come\c{c}ando com -1, e onde -1 \'{e} o \'{u}ltimo elemento da sequ\^{e}ncia, -2 \'{e} o pen\'{u}ltimo elemento da sequ\^{e}ncia, e assim sucessivamente.
      \end{description}

     \begin{lstlisting}
    >>> # Indexando strings
    >>> pyStr = "Programando"
    >>> print (len(pyStr))
    11
    >>> print (pyStr)
    Brasil verde amarelo
        \end{lstlisting}




      \item \textit{Operador de Fatias}\\
      O operador de acesso a itens (caracteres individuais) tamb\'{e}m pode ser utilizado como operador de fatias, para extrair uma fatia inteira (subsequ\^{e}ncia) de caracteres de um string. O operador de Fatias possui tr\^{e}s sintaxes:\\
      \texttt{seq[ inicio ]}\\
      \texttt{seq[ inicio : fim ]} \\
      \texttt{seq[ in\'{\i}cio : fim : step ]}\\
      onde \texttt{in\'{\i}cio, fim} e \texttt{step} s\~{a}o n\'{u}meros inteiros.
     \begin{lstlisting}
    >>> # Indexando strings
    >>> pyStr = "Programando Python"
    >>> print (len(pyStr))
    11
    >>> print (pyStr)
    Brasil verde amarelo
        \end{lstlisting}

    \end{itemize}

     %%%%%%%%=================================
    \section{Tipos de Dados de Cole\c{c}\~{a}o}
    %%%%%%%%=================================


     %%%........................
            \subsection{Tipos Sequenciais}
     %%%........................


     %%%........................
            \subsection{Tipos Conjunto}
     %%%........................


     %%%........................
            \subsection{Tipos Mapeamento}
     %%%........................




    %%%%%%%%=================================
    \section{Estrutura de Controle e Fun\c{c}\~{o}es}
    %%%%%%%%=================================

     %%%........................
            \subsection{O comando IF}
     %%%........................


      %%%........................
            \subsection{La\c{c}o FOR}
     %%%........................

     %%%........................
            \subsection{La\c{c}o WHILE}
     %%%........................


    %%%%%%%%======================
    \section{M\'{o}dulos e pacotes}
    %%%%%%%%======================



       %%%........................
            \subsection{M\'{o}dulos}
     %%%........................



          %%%........................
            \subsection{Pacotes}
     %%%........................






    C\'{o}digo fonte para a linguagem Python:
    \begin{lstlisting}
    number_1 = int(input('Ingresse o primeiro numero: '))
    number_2 = int(input('Ingresse o segundo numero: '))

    # Soma
    print('{} + {} = '.format(number_1, number_2))
    print(number_1 + number_2)

    # Substra\c{c}\~{a}o
    print('{} - {} = '.format(number_1, number_2))
    print(number_1 - number_2)

    # Multiplica\c{c}\~{a}o
    print('{} * {} = '.format(number_1, number_2))
    print(number_1 * number_2)

    # Divis\~{a}o
    print('{} / {} = '.format(number_1, number_2))
    print(number_1 / number_2)
    \end{lstlisting}





